\documentclass[	
  noindent
]{elteikthesis}[2024/04/26]

\usepackage{booktabs}

\begin{document}

\documentlang{hungarian}

\begin{titlepage}
    \centering
    \vspace*{3cm}
    {\Huge\bfseries Az elkészített szoftverek \par}
    \vspace{2cm}
    {\Large Sándor Balázs\par}
    \vspace{1cm}
    {\large 2025. április 24. \par}
    \vfill
    {\includegraphics[width=0.2\textwidth]{images/elte_cimer_szines.eps}\par}
\end{titlepage}

\tableofcontents

\chapter{Elkészített szoftverek}

    \section{A melléklet célja}
    A diplomamunka során elkészített szoftverek központi szerepet játszottak az adatfeldolgozásban, a tanítási és tesztelési folyamatok automatizálásában, valamint a mért adatok kiértékelésében. A teljes munkafolyamat Python nyelven készült, Jupyter notebookok, PyTorch keretrendszer, valamint a matplotlib, seaborn, pandas és numpy könyvtárak használatával. A modellek betanítása és kiértékelése saját fejlesztésű kódbázison történt, amely lehetővé tette az egyedi igények szerinti metrika- és grafikon-előállítást.

    A fejlesztett szoftvermodulok a következő főbb komponensekre bonthatók:

        Adat-előkészítő és konvertáló eszközök: Ezek a programrészek a nyers fotóállományból készítettek előfeldolgozott, egységesített és annotált képadatbázist. Ide tartozik az EXIF-adatok megőrzését biztosító átalakítás, az MD hash alapú elnevezés és ismétlődés szűrés, az automatikus kategorizálás, a megfelelő képméretre vágás, tömörítés, valamint a tanító/tesztelő/validáló halmazok létrehozása.

        A modellek tanítását vezérlő szkriptek: A tanítórendszer lehetővé teszi különféle modellek (ResNet50, DenseNet121, EfficientNet-B0, ConvNeXt-Tiny, stb.) betöltését, finomhangolását, valamint többféle tanítási fázis (pl. teljes tanítóhalmaz, iteratív bővítés) automatizálását. A szkriptek rögzítik az összes fontos tanítási metrikát (loss, accuracy, precision, recall, F1-score), epochonként és osztályszinten.

        Tesztelési és kiértékelő modulok: Ezek a programok lehetővé teszik a betanított vagy gyári súlyokkal rendelkező modellek teljesítményének összehasonlítását különféle metrikák alapján, mind Top-1, Top-5, mind osztályspecifikus pontosság szerint. A program automatikusan készít confusion matrixokat, osztályeloszlás-görbéket, metrika-idő vagy metrika-mintaszám függvényeket.

        Vizualizációs alrendszer: A metrikák kiértékelését követően a rendszer képes automatikusan egységes formátumú és jól olvasható grafikonokat generálni. A matplotlib és seaborn könyvtárakkal készült diagramok a tanítási folyamat vizuális dokumentációját nyújtják, megkönnyítve az összehasonlítást.

        Egyedi fejlesztésű kis modell és fejrész prototípusok: A szoftvercsomag része az erőforrás-kímélő, ResNet18-alapú lokális modell fejlesztését támogató kódbázis is, amely lehetővé teszi különféle hálózati „fejek” tesztelését és kiértékelését. Ezek külön paraméterezhetőek és egységes módon mérhetőek a többi modellhez hasonlóan.

    Minden programrész jól dokumentált, verziókövetett formában készült, a főbb beállítások és paraméterek külön konfigurációs blokkokban találhatóak a notebook-okban. A virtualizált környezet kialakításáért felelős telepítő blokkok is elhelyezésre kerültek a notebook-okba a könnyebb megismételhetőség érdekében. A melléklet tartalmazza a szoftverrendszer logikai felépítésének leírását, a legfontosabb fájlok magyarázatát, valamint a kód működésének szemléltetését példákkal.

    Az elkészített programkódok a projekt GitHub könyvtárában találhatóak. 

    A projekt GitHub repository-ja:

    \url{https://github.com/SandorBalazsHU/elte-ik-msc-thesis}

    \section{Képkonverzió}
      A képkonverziót egy saját fejlesztésű, \texttt{convert.bat + convert\_runner.py} nevű szkript végezte, amely Windows batch és Python alapon működő, multimappás kötegelt feldolgozásra képes rendszerként lett kialakítva. Feladata a képek egységes átméretezése \texttt{224×224} pixelre, valamint megfelelő tömörítéssel történő konvertálása \texttt{JPG} formátumba.
      
      A folyamat első lépéseként a szkript listázza az adott mappában található képfájlokat, majd minden képre külön feldolgozási folyamatot indít, két lépésben:
      
      \begin{itemize}
        \item Az \textbf{IrfanView} eszközzel történik a nyers képfájlok (pl. NEF formátum) konvertálása és méretezése.
        \item Ezt követően az \textbf{ImageMagick} segítségével történik a középpontos, 1:1 oldalarányú képkivágás.
      \end{itemize}
      
      A két eszköz kombinálására azért volt szükség, mert:
      \begin{itemize}
        \item Az \texttt{ImageMagick} nem minden esetben kezelte megbízhatóan a \texttt{NEF} formátumot,
        \item Az \texttt{IrfanView} viszont nem biztosított precíz középpontos kivágást.
      \end{itemize}
      
      A szkript futtatását egy Python program végezte, amely rekurzívan indította el a \texttt{convert.bat} szkriptet minden egyes almappára, lehetővé téve ezzel a több szálon futó, párhuzamos konverziót.
      
      A rendszer teljesítménye rendkívül hatékonynak bizonyult: a többmagos CPU-teljesítményt teljes mértékben kihasználta, és magas képfeldolgozási sebességet ért el. Az alábbi táblázatok részletesen bemutatják az egyes műveletek időigényét és a teljes rendszer átlagos teljesítményét.
      
      \vspace{1em}
      \begin{table}[H]
      \centering
      \caption{Képkonverziós műveletek időigénye és optimalizálhatósága}
      \begin{tabular}{|l|c|l|}
      \hline
      \textbf{Művelet} & \textbf{Átlagos idő} & \textbf{Lehetőség a gyorsításra?} \\
      \hline
      NEF $\rightarrow$ JPG konverzió & $\sim$35–45 ms & IrfanView optimalizálva \\
      Átméretezés & $\sim$10–20 ms & ImageMagick CPU-teljesítményen \\
      MD5 hash / átnevezés & $<$5 ms & Pythonban villámgyors \\
      Kézi overhead & $\sim$2–3 s / futás & Teljesen automatizálható \\
      \hline
      \end{tabular}
      \end{table}
      
      \vspace{1em}
      \begin{table}[H]
      \centering
      \caption{Átlagos feldolgozási teljesítmény}
      \begin{tabular}{|l|c|}
      \hline
      \textbf{Sebességmutató} & \textbf{Érték} \\
      \hline
      Képenkénti feldolgozási idő & \textbf{52,6 ms} \\
      Átlagos képfeldolgozási sebesség & \textbf{$\sim$19,0 kép / másodperc} \\
      Konverziós sávszélesség (becsült) & \textbf{$\sim$1,27 MB/s} (65 KB/kép mellett) \\
      10 000 kép becsült ideje & \textbf{$\sim$10,4 perc} \\
      150 000 kép becsült ideje & \textbf{$\sim$2 óra 36 perc} \\
      \hline
      \end{tabular}
      \end{table}
      
      \vspace{1em}
      
      A következő ábra egy mappa feldolgozását mutatja be, amely során a \texttt{convert.bat} és a \texttt{convert\_runner.py} működött együtt a konverziós feladatok elvégzésére:
      
      \begin{figure}[H]
        \centering
        \includegraphics[width=1.0\textwidth]{images/db/scripts/convert-bat.png}
        \caption{A \texttt{convert.bat} és \texttt{convert\_runner.py} futása egy mappa feldolgozására.}
        \label{fig:01_convert-bat}
      \end{figure}
    
    \subsection{convert.bat képkonverziós program.}
    \lstset{caption={convert.bat párhuzamos egy mappán belüli képkonverziós szkript.}, label=src:convert-bat}
    \begin{lstlisting}[language={[x86masm]Assembler}]
@echo off
setlocal enabledelayedexpansion

:: -------------------- Settings --------------------
set "iview=C:\Program Files\IrfanView\i_view64.exe"
set "magick=magick"
set "inputFolder=%~dp0"
set "outputFolder=%inputFolder%resized"

:: -------------------- Create output folder --------------------
if not exist "%outputFolder%" (
    echo [INFO] Creating 'resized' folder...
    mkdir "%outputFolder%"
)

:: -------------------- Count input files --------------------
echo.
echo [STEP 1] Counting input files...

set /a count=0
for %%f in ("%inputFolder%*.nef") do set /a count+=1
for %%f in ("%inputFolder%*.jpg") do set /a count+=1

echo Found !count! images to process.

:: -------------------- Resize in parallel --------------------
echo.
echo [STEP 2] Converting all to resized JPG (short side = 224 px)...

set /a i=0
for %%f in ("%inputFolder%*.nef") do (
    set /a i+=1
    start /b "" "%iview%" "%%f" /resize_short=224 /aspectratio /resample /jpgq=80 /convert="%outputFolder%\%%~nf_nef.jpg"
    call :progress !i! !count! "Converting"
)

for %%f in ("%inputFolder%*.jpg") do (
    set /a i+=1
    start /b "" "%iview%" "%%f" /resize_short=224 /aspectratio /resample /jpgq=80 /convert="%outputFolder%\%%~nf_jpg.jpg"
    call :progress !i! !count! "Converting"
)

echo.
echo Waiting for all IrfanView conversions to finish...
call :wait_irfan

:: -------------------- Crop in parallel --------------------
echo.
echo [STEP 3] Cropping to center 224x224 (parallel)...

set /a totalCrop=0
for %%f in ("%outputFolder%\*.jpg") do (
    set /a totalCrop+=1
)

set /a i=0
for %%f in ("%outputFolder%\*.jpg") do (
    set /a i+=1
    start /b "" cmd /c %magick% "%%f" -gravity center -crop 224x224+0+0 +repage "%%f"
    call :progress !i! !totalCrop! "Cropping"
)

echo.
echo Waiting for all cropping to finish...
call :wait_magick

echo.
echo [DONE] All images processed to 224x224 JPG in 'resized' folder.
endlocal
pause
exit /b

:: -------------------- Progress display --------------------
:progress
setlocal
set "current=%~1"
set "total=%~2"
set "label=%~3"
set /a percent=(current*100)/total
echo [%label%] !percent!%% (!current! / !total!)
endlocal
exit /b

:: -------------------- Wait for IrfanView jobs to finish --------------------
:wait_irfan
tasklist | find /I "i_view64.exe" >nul
if not errorlevel 1 (
    timeout /t 1 >nul
    goto wait_irfan
)
exit /b

:: -------------------- Wait for ImageMagick jobs to finish --------------------
:wait_magick
tasklist | find /I "magick.exe" >nul
if not errorlevel 1 (
    timeout /t 1 >nul
    goto wait_magick
)
exit /b
    \end{lstlisting}

    \subsection{convert-runner.py párhuzamos többmappás képkonvertáló.}
    \lstset{caption={convert-runner.py párhuzamos sokmappás képkonverziós szkript.}, label=src:convert-runner-bat}
    \begin{lstlisting}[language={Python}]
import os
import shutil
import subprocess
import sys

def main():
    # A szkript konyvtara
    base_dir = os.path.dirname(os.path.abspath(__file__))
    convert_bat = os.path.join(base_dir, 'convert.bat')

    if not os.path.isfile(convert_bat):
        print(f"[ERROR] 'convert.bat' nincs megtalalhato a {base_dir} konyvtarban.")
        sys.exit(1)

    # 1) Mappaterkep keszitese (elozetesen, hogy a kesobbi 'resized' mappakat ne talalja meg)
    directories = []
    for root, dirs, _ in os.walk(base_dir):
        # Ha a mappa neve 'resized', ugorjuk at
        if os.path.basename(root).lower() == 'resized':
            continue
        directories.append(root)

    print(f"[INFO] Felderitett {len(directories)} konyvtarat.")

    # 2) convert.bat masolasa minden mappaba
    for d in directories:
        dest = os.path.join(d, 'convert.bat')
        try:
            shutil.copy2(convert_bat, dest)
            print(f"[COPY] {dest}")
        except Exception as e:
            print(f"[ERROR] Nem sikerult masolni ide: {d} ({e})")

    # 3) A batch fajl futtatasa minden mappaban
    for d in directories:
        bat_path = os.path.join(d, 'convert.bat')
        print(f"[RUN] {bat_path}")
        # Windows-on futtatashoz
        subprocess.run(['cmd', '/c', bat_path], cwd=d)

if __name__ == '__main__':
    main()      
    \end{lstlisting}

  \section{Átnevezés és ismétlődések szűrése}
    A fájlok MD5 hash alapján történő átnevezése kulcsfontosságú lépés volt az adatbázis egységességének és redundancia-mentességének biztosítása érdekében. Mivel minden egyes már konvertált és átméretezett kép a saját MD5 hash értékét kapta fájlnévként, gyakorlatilag kizárhatóvá vált, hogy két azonos kép duplikáltan szerepeljen az adatbázisban.
    
    Ez a művelet azonban a nagy fájlszám és a sok mappa miatt nem volt triviális, és egy hatékony, párhuzamos feldolgozásra képes szkript megírását igényelte. E célra két Python szkriptet fejlesztettem: \texttt{rename.py} és \texttt{rename2.py}. Ezek a programok rekurzívan végigjárták a teljes mappaszerkezetet, és 50 képenként új szálat indítottak, így biztosítva a feldolgozás párhuzamosságát. Minden szál az adott képcsoportot hash-elte, majd az eredmény alapján nevezte át őket.
    
    A megközelítés jelentős sebességnövekedést eredményezett, bár még így is kissé lassabbnak bizonyult, mint a konverziós-átméretező szkript, amely a rendszer egyik leggyorsabb komponensének bizonyult.
    
    Egy átnevezési példa:
    
    \begin{center}
    \texttt{DSC0123.jpg} $\rightarrow$ \texttt{9f6e3148b22b4e13a7b52b66f99fd912.jpg}
    \end{center}
  
  \subsubsection{Az átnevezési szkript futása}
  
  A következő ábra egy konkrét futtatást mutat be, ahol a \texttt{rename.py} szkript egy mappa képeit hash-eli és nevezi át párhuzamos szálakon keresztül:

  \begin{figure}[H]
    \centering
    \includegraphics[width=1.0\textwidth]{images/db/scripts/rename-py-1.png}
    \caption{A \texttt{rename.py} szkript futása.}
    \label{fig:02-rename.py}
  \end{figure}

  \begin{figure}[H]
      \centering
      \includegraphics[width=1.0\textwidth]{images/db/scripts/rename-py.png}
      \caption{A \texttt{rename.py} szkript futása.}
      \label{fig:02-rename.py}
  \end{figure}
  
  \subsection{A rename2.py MD5 átnevező szkript.}
    \lstset{caption={A rename2.py MD5 átnevező szkript.}, label=src:rename2.py}
    \begin{lstlisting}[language={Python}]
import os
import random
import string
import hashlib
from tqdm import tqdm

# --- Beallitasok ---
NAME_LENGTH = 25                        # Valtoztathato nevhossz (pl. 20-25)
USE_MD5 = True                          # Ha True, az uj nev az MD5 hash lesz
TARGET_EXTENSIONS = {'.jpg', '.jpeg', '.png'}  # Feldolgozando kiterjesztesek

# A szkript konyvtara lesz az alapertelmezett gyoker
BASE_DIR = os.path.dirname(os.path.abspath(__file__))

# --- Random nev generalasa ---
def generate_random_name(length):
    chars = string.ascii_letters + string.digits
    return ''.join(random.choices(chars, k=length))

# --- MD5 hash generalasa ---
def generate_md5(filepath):
    with open(filepath, 'rb') as f:
        return hashlib.md5(f.read()).hexdigest()

# --- Konyvtarak felderitese ---
directories = []
for root, dirs, _ in os.walk(BASE_DIR):
    directories.append(root)

print(f"[INFO] Felderitett konyvtarak szama: {len(directories)}")
for d in directories:
    print(f"  - {os.path.relpath(d, BASE_DIR)}")

# --- Minden konyvtar feldolgozasa ---
for FOLDER in directories:
    # Fajlok listazasa a celkiterjesztesekkel
    files = [
        f for f in os.listdir(FOLDER)
        if os.path.isfile(os.path.join(FOLDER, f))
            and os.path.splitext(f)[1].lower() in TARGET_EXTENSIONS
    ]

    if not files:
        print(f"[INFO] {os.path.relpath(FOLDER, BASE_DIR)}: nincs megfelelo fajl.")
        continue

    print(f"[INFO] {os.path.relpath(FOLDER, BASE_DIR)}: {len(files)} fajl atnevezese...")

    for filename in tqdm(files, desc=f"Renaming in {os.path.relpath(FOLDER, BASE_DIR)}"):
        full_path = os.path.join(FOLDER, filename)
        ext = os.path.splitext(filename)[1].lower()

        # uj nev generalasa
        if USE_MD5:
            newname = generate_md5(full_path)
        else:
            newname = generate_random_name(NAME_LENGTH)

        new_filename = f"{newname}{ext}"
        new_path = os.path.join(FOLDER, new_filename)

        # utkozes elkerulese
        while os.path.exists(new_path):
            if USE_MD5:
                newname += '_1'
            else:
                newname = generate_random_name(NAME_LENGTH)
            new_filename = f"{newname}{ext}"
            new_path = os.path.join(FOLDER, new_filename)

        os.rename(full_path, new_path)

    print(f"[DONE] {os.path.relpath(FOLDER, BASE_DIR)}: minden fajl atnevezve.")

print("[ALL DONE] Az osszes konyvtarban vegezve.") 
    \end{lstlisting}

  \section{Az elkészült képek összemásolása}
  A képfeldolgozó pipeline utolsó lépése az összes .jpg fájl egy mappába másolása amit a sumcopy.py szkript végzett el amiben egy utolsó ismétlődés szűrő mechanizmus is dolgozott.

  \begin{figure}[H]
      \centering
      \includegraphics[width=1.0\textwidth]{images/db/scripts/sumcopy-py.png}
      \caption{A \texttt{sumcopy.py} szkript futása.}
      \label{fig:03-sumcopy.py}
  \end{figure}

  \subsection{sumcopy.py - Az elkészült képek összemásolása}
  \lstset{caption={A sumcopy.py - Az elkészült képek összemásolása.}, label=src:sumcopy.py}
  \begin{lstlisting}[language={Python}]
import os
import shutil
import sys

def main():
    # A szkript konyvtara lesz a gyoker
    base_dir = os.path.dirname(os.path.abspath(__file__))
    
    # Celmappa a full_resized
    full_resized_dir = os.path.join(base_dir, 'full_resized')
    if not os.path.exists(full_resized_dir):
        os.makedirs(full_resized_dir)
        print(f"[INFO] Letrehozva: {full_resized_dir}")
    else:
        print(f"[INFO] Mar letezik: {full_resized_dir}")

    # 1) Keressuk meg az osszes 'resized' mappat
    resized_dirs = []
    for root, dirs, files in os.walk(base_dir):
        # ha a root mappa neve 'resized', vegyuk fel
        if os.path.basename(root).lower() == 'resized':
            resized_dirs.append(root)

    print(f"[INFO] {len(resized_dirs)} darab 'resized' mappa talalhato.")

    # 2) Masolas full_resized-be, nevutkozesnel kihagyjuk
    for rd in resized_dirs:
        print(f"[PROCESS] {rd} tartalmanak masolasa...")
        for fname in os.listdir(rd):
            src = os.path.join(rd, fname)
            dst = os.path.join(full_resized_dir, fname)

            # csak fajlokat masolunk
            if not os.path.isfile(src):
                continue

            if os.path.exists(dst):
                print(f"  [SKIP] Mar letezik: {fname}")
                continue

            try:
                shutil.copy2(src, dst)
                print(f"  [COPY] {fname}")
            except Exception as e:
                print(f"  [ERROR] Nem sikerult masolni {fname}: {e}")

    print("[DONE] Minden 'resized' mappa feldolgozva.")

if __name__ == '__main__':
    main()
  \end{lstlisting}

  \section{A további kódok a GitHub repository-ban találhatóak.}

%  \subsection{}
%  \lstset{caption={}, label=src:sumcopy.py}
%  \begin{lstlisting}[language={Python}]
%  \end{lstlisting}

\listoffigures
\listoftables
\end{document}